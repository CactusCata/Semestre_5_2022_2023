\documentclass[10pt, a4paper, parskip=full]{scrartcl}
%	PREAMBULE

% Packages
\usepackage[utf8]{inputenc} % encodage utf-8
\usepackage[T1]{fontenc}
\usepackage[french]{babel} % traduit chapter en chapitre
\usepackage{textcomp}
\usepackage{amsmath,amssymb}
\usepackage{amsthm}
\usepackage{lmodern}
\usepackage{graphicx} % permet d'ajouter des images
\usepackage[dvipsnames,svgnames]{xcolor}
\usepackage{microtype}
\usepackage{lipsum}
\usepackage{graphicx} % pour ajouter 
\usepackage{hyperref} % permet de faire des liens href
\hypersetup{colorlinks=true,linkcolor=Brown,urlcolor=Blue,breaklinks=true,bookmarks
=true,pdfstartview=XYZ} % on redefinit les options pour les href

\newcommand{\macommand}{ecris des trucs cools}

% si on veut ajouter des espaces dans le texte des fonctions
% on peut faire:
\newcommand{\macommandd}{ecris des trucs cools\xspace}

% On peut ajouter des arguments:
\newcommand{\macommanddd}[1]{ma commande: "#1" !\xspace}

\newcommand{\nompropre}[2]{#1 \textsc{#2}\xspace}

% Affectactions au variables (title, author et date)
\title{Mon 1\ier{} document}
\author{Adam Chareyre}
%\date{} Date mise à la date de la compilation

\begin{document}
%Corps du document

% Affiche le contenu des variables title, author et date
\maketitle

% Abstract
\begin{abstract}
\lipsum[1-2]
\end{abstract}

% Table des matieres
\tableofcontents

\section{Titre de la section}

% ou
% \section*{Titre de la section} pour ne pas mettre de numero
% mais il n'apparaitra pas dans la table des matières, pour l'ajouter, il faut ajouter la ligne:
% \addsec{Titre de la section}


Mon premier document \LaTeX   dsd

\lipsum

{1\ier} document: \href{http://google.com}{truc}

\includegraphics[trim=1cm 1cm 1cm 1cm,clip]{img.jpeg}
% mettre clip quand l'image se place sur du texte

% OU

% L'image se placera où elle veut
% on peut lui demander une position idéale, par exemple t, p, etc
\begin{figure}
\centering
\includegraphics[trim=1cm 1cm 1cm 1cm,clip]{img.jpeg}
\caption{Titre de la figure}\label{img.jpeg}
\end{figure}

\begin{center} % permet de centrer la figure
\begin{tabular}{|c|c|c|} % positionne à droite/gauche/milieu de la colonne
\hline % permet d'ajouter des lignes horizontales
a b	& c d & e \\
\hline % permet d'ajouter des lignes horizontales
f g	& h i & j \\ % le "\\" permet d'ajouter le dernier \hline sans erreur
\hline
\end{tabular}
\end{center}

\macommand{}
\macommand{}
\macommandd{}
\macommandd{}

\macommanddd{lol}

\nompropre{adam}{                                                                                                                                                   chareyre}

\cite{lcfr}

\bibliographystyle{plain}
\bibliography{biblio}


\end{document}