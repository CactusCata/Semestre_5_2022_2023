\documentclass[10pt, a4paper, parskip=full]{scrartcl}
%	PREAMBULE

% Packages
\usepackage[utf8]{inputenc} % encodage utf-8
\usepackage[T1]{fontenc}
\usepackage[french]{babel} % traduit chapter en chapitre
\usepackage{textcomp}
\usepackage{amsmath,amssymb}
\usepackage{lmodern}
\usepackage{graphicx} % permet d'ajouter des images
\usepackage[dvipsnames,svgnames]{xcolor}
\usepackage{microtype}
\usepackage{lipsum}
\usepackage{showkeys} % permet de debugger les labels
\usepackage{hyperref} % permet de faire des liens href et ref (pour les label et les URL)
\hypersetup{colorlinks=true,linkcolor=Brown,urlcolor=Blue,breaklinks=true,bookmarks
=true,pdfstartview=XYZ} % on redefinit les options pour les href et les ref

% Affectactions au variables (title, author et date)
\title{Équations différentielles}
\author{Gloria Faccanoni}
\date{15 janvier 2014}

\begin{document}
%Corps du document

% Affiche le contenu des variables title, author et date
\maketitle

% Abstract
\begin{abstract}
\lipsum[1-2]
\end{abstract}

% Table des matieres
\tableofcontents

\section*{Introduction}
\addsec{Introduction}
\lipsum[1-2]

\section{Rappels}\label{sec.rappels}
\lipsum[1-2]

\subsection{Condition initiale}
\lipsum[1-2]

\subsection{Problème de Cauchy}
\lipsum[1-2]

\section{Exercices}
\lipsum[1-2]
Voir à la section~\ref{sec.rappels} à la page \pageref{sec.rappels}.




\end{document}