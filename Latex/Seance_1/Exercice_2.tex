\documentclass[10pt, a4paper, parskip=full]{scrartcl}
%	PREAMBULE

% Packages
\usepackage[utf8]{inputenc} % encodage utf-8
\usepackage[T1]{fontenc}
\usepackage[french]{babel} % traduit chapter en chapitre
\usepackage{textcomp}
\usepackage{amsmath,amssymb}
\usepackage{lmodern}
\usepackage{graphicx} % permet d'ajouter des images
\usepackage[dvipsnames,svgnames]{xcolor}
\usepackage{microtype}
\usepackage{lipsum}
\usepackage{showkeys} % permet de debugger les labels
\usepackage{hyperref} % permet de faire des liens href et ref (pour les label et les URL)
\hypersetup{colorlinks=true,linkcolor=Brown,urlcolor=Blue,breaklinks=true,bookmarks
=true,pdfstartview=XYZ} % on redefinit les options pour les href et les ref
\usepackage[pointedenum]{paralist} % C'est cool !!!! permet de mettre des 1.1.2 dans les listes, plus beau

% Affectactions au variables (title, author et date)
\title{Équations différentielles}
\author{Gloria Faccanoni}
\date{15 janvier 2014}

\begin{document}
%Corps du document

% Affiche le contenu des variables title, author et date
\maketitle

% Abstract
\begin{abstract}
\lipsum[1-2]
\end{abstract}

% Table des matieres
\tableofcontents

\begin{enumerate}

	\item Ont des ailes
	\begin{enumerate}
		\item Volent
		\begin{enumerate}
			\item Hirondelle
			\item Aigle
			\item Ara
		\end{enumerate}
		\item Ne volent pas
		\begin{enumerate}
			\item Poule
			\item Autruche
		\end{enumerate}
		
	\end{enumerate}
	
	\item Vivent dans la mer
	\begin{itemize}
		\item Respirent dans l'eau
		\begin{itemize}
			\item Requin
			\item Hippocampe
		\end{itemize}
		\item Respirent hors de l'eau
		\begin{itemize}
			\item Baleine
			\item dauphin
		\end{itemize}
	\end{itemize}
	
	\item Vivent sur terre
	\begin{description}
		\item[Australie:] kangourou, koala
		\item[Europe:] loup
	\end{description}
\end{enumerate}




\end{document}